%------------------------------------------------------------------------------
\chapter{cppmake\index{cppmake}}
\label{sec:share000.cppmake}
%------------------------------------------------------------------------------
\section{Introduction}
\label{sec:share000.cppmake.introduction}
%------------------------------------------------------------------------------

cppmake allows expressing build dependencies in C++. A build policy then takes
care of building targets, or rejecting a build, when the sources are not conform
the policy.

cppmake, is a library written in standard C++, with limited platform specific
code. Also the policy is written in C++. The policy can be adjusted to conform
the needs of a specific project. Choosing for a standard C++ solution assures that:

\begin{itemize}
\item C++ programmers do not need to learn an additional language for expressing 
      their build expressions
\item C++ programmers are able to debug the make process. Please note that (almost)
      none of the other build languages provide debugging facilities
\item Build, product and project policies (which exceed standard build expressions)
      can be expressed naturally in C++. Think about running class tests, deploying
	  software, static code checking, promotion, reporting etc.
\item Portability
\end{itemize}

Although using the development language as a make language is an almost\footnote{
The tool cons is written in Perl and uses Perl scripts as a make language for building
Perl projects (and more). The tool scons is written in Python and uses Python scripts 
as a make language for building Python projects (and more).}
unique idea, this is not the major ambition of cppmake. The major ambition is: 
'providing a safe system for incremental builds'. The fact that any build tool I've 
worked with so far could not guarantee a safe incremental build, resulting in the 
need for complete rebuilds. And even those were often not safe either; not safe in 
the sense that the output was not deterministic.

%------------------------------------------------------------------------------
\section{Safe incremental builds and policy}
\label{sec:share000.cppmake.safe_incremental_builds}
%------------------------------------------------------------------------------

Having safe incremental builds require a specific build policy. When it is allowed
to do everything that is not explicitly forbidden by the C++ standard, then a
safe incremental build is not possible. As C++ allows so much, because the language
focusses so much on performance over safety, one could develop countless policies
that fullfill the `safe incremental build' ambition to a desired extend.

Respecting C++' fundamental choice to focus on performance, it is also not the
intention for cppmake to define one specific policy. Instead, projects may develop
their own build policy that fits their needs best. On the other hand !share000!
does provide only one policy. So when a product owner chooses another policy, it
will be under e.g. !share001!.

The !share000! policy takes a data driven approach to allow safe incremental builds.
The !share000! policy relates to the following data:
\begin{enumerate}
\item Specific data files read and written by cppmake
\item Sources from the archive (git)
\item Artifacts, created upon the initiative of cppmake. (in !.../procts/*/tgt! and 
      !.../procts/*/obj!)
\end{enumerate}

It seems as if tools (like compiler and linker) are missing, but these tools are
supposed all to be present in the archive under version control.

The specific data files for cppmake are binary, and the developer is supposed not
to modify the content. When any of these files are deleted, then cppmake will do
a rebuild all afterall. In contrast to other build tools, the developer should also
not modify any file in !.../procts/*/tgt! and !.../procts/*/obj!. cppmake will check
whether any file was modified. In case of any violation of this rule, cppmake will
deliberatly punish the developer by enforcing a full build. In this way we hope
that the developer learns not to touch !.../procts/*/tgt! and !.../procts/*/obj!
files.

This policy assures cppmake has sole control over all files that participate in
building the targets. Items 1 and 3 are not to be touched by anyone or anything
except cppmake, item 2 is under source control, being git. cppmake will simply
depend on git in finding out which sources have changed compared to the former
build.

The data oriented approach also implies that when the developer issues a build,
but the sources of a failing unit were not changed, then cppmake will not build
the unit again, because cppmake is convinced that the result shall be the same.
Instead, cppmake will present the errors and warning of the previous build.
To realize this, cppmake will store all errors and warnings for any last build
of a unit.

Please note that the data oriented approach is not sufficient for safe incremental
builds. Additionally, a correct dependency graph is to be maintained, that provides
the information to decide what has to be rebuild.

%------------------------------------------------------------------------------
\section{Design overview}
\label{sec:share000.cppmake.design_overview}
%------------------------------------------------------------------------------

See figure~\ref{fig:share000.cppmake.overview} for an overview of the cppmake
design. The above mentioned !policy! takes a central position. It is the hope and
intention that when other product owners formulate another policy, that they only
need to change the !policy! component.

\begin{figure}[tbh]
    \begin{center}
        \includegraphics{\procts share000/comp/doc_cppmake/class_overview.eps}
    \end{center}
    \caption{Overview (UML class diagram) }
	\label{fig:share000.cppmake.overview}
\end{figure}

Most components surrounding the !policy! (!depgraph!, !target!, !path!, !ide_filter!
!makeables!, !process! and !executor!) take up a specific responsibility of which
is currently expected that it will be needed for any policy. So, it is the hope and
intention that when other product owners formulate another policy, that they do
{\em not} need to change those components.

A rather recognizable component is the !makefile!. The !makefile! will contain the
build or dependency instruction provided by the developer. As indicated earlier,
the !makefile! is also a C++ file, and whenever it is changed it needs to be 
recompiled. cppmake shall take care of that. After building the !makefiles!s 
(typically you'll find multiple !makefiles!s), 9 components at the top of
figure~\ref{fig:share000.cppmake.overview} will be linked into a !.dll! or !.so!.

The !make.exe! component is responsible for loading the !.dll! or !.so! and starting
the build. All build knowledge shall be in the !.dll! or !.so!; !make.exe! is only
a platform for executing such a build from the !.dll! or !.so!.

%------------------------------------------------------------------------------
\section{Make.exe}
\label{sec:share000.cppmake.make.exe}
%------------------------------------------------------------------------------

The responsibilities of !make.exe! are: 

\begin{enumerate}
\item Copy the !makefile.dll! to the procts directory
\item Assure that only one !cppmake.exe! is running within projects
\item Dynamically link to two functions of !makefile.dll!:
\begin{itemize}
   \item !execute()!
   \item !abort_execution()!
\end{itemize}
\item Call !execute()! while passing the commandline parameters 
\item In case of a control-C, call !abort_execution!
\item In case !execute()! returns 1, start all over from item 1.
    This allows !makefile.dll! to build itself and after being called again it builds the targets
\item Print "Build succeeded" when the build was succesfull and "Build failed" otherwise
\item Return 0 when the build was succesfull and -1 otherwise
\end{enumerate}

Implementation of !cppmake.exe! is (unfortunatelly) platform dependent.
For the user, cppmake behaves identical on any platform (just like git and LaTeX)
Implementation of !makefile.dll! is intended to be as much standard C++ as possible.

!cppmake.exe! shall be build by `flying start'. Due to the limited responsibility
of !cppmake.exe!, the limited size of its source, it is not expected that !cppmake.exe!
needs a rebuild during normal development.

%------------------------------------------------------------------------------
\section{Makefile}
\label{sec:share000.cppmake.makefile}
%------------------------------------------------------------------------------

The !makefile!s contains a set of directives used by the cppmake tool to generate
a targets. At link-time, the !makefile! depends on !cppmake.lib!; at compile time
!makefile! depends on !cppmake/makefile.h!.

All !makefile!s in one stem repository are compiled and linked into !makefile.dll!
or !makefile.so!.
This !makefile.dll! or !makefile.so! is picked up by the !make.exe! tool.

A single !makefile! has typically a construction as shown in 
listing~\ref{lst:share000.cppmake.makefile}. By the object !main! is in fact a
static object which will be constructed when !makefile.dll! or !makefile.so! are
loaded. By inheriting from !cppmake::main_t!, all such static objects are added
to a global singleton list. To realize the build, all objects in that list will
be visited by calling overloaded virtual functions.

\begin{lstlisting}[
    float,
    frame=trBL,
    caption={The make object},
    label={lst:share000.cppmake.makefile},
    gobble=4]
	namespace { // anonymous

	class main_t
		: cppmake::main_t
	{
		...
	} main;

	}; // namespace anonymous
\end{lstlisting}

Note that it is no problem that all these static object have the same type name
(!main_t!) and object name (!main!), because all these objects are in an anonymous
namespace. A single object is intended for a single target. Note that it is also
possible to have multiple objects derived from !cppmake::main_t! in the same
!makefile.cpp!. In that case multiple targets will be build. Of coarse, the type
and object name must differ make objects for multi target make files.

%------------------------------------------------------------------------------
\section{Executor}
\label{sec:share000.cppmake.executor}
%------------------------------------------------------------------------------

The executor is responsible for:
\begin{description}
\item[\ix{RT.executor.create.policy}]
    Create the policy and pass the commandline arguements
\item[\ix{RT.executor.build}]
    In case a build needs to be performed:
    \begin{itemize}
    \item start the processes which are supplied by the dependency graph
    \item report them complete when a process is finished
    \item continue doing this until no more processes are supplied
    \end{itemize}
\item[\ix{RT.executor.build.threading}]
    Handle all threading threading issues. This includes
    presenting a thread safe print line so that policy can print on the console
\item[\ix{RT.executor.build.abort}]
    Handle user abort by aborting all running processes. Finish book keeping in a controlled way.
\item[\ix{RT.executor.exceptions}]
    Catching all exceptions and pass them on to policy when possible
\item[\ix{RT.executor.return}]
    Return:
    \begin{itemize}
    \item  0 when build finished succefully and no rerun is needed, 
    \item  1 when build finished succefully and rerun is needed,
    \item -1 when the build was not successfull
    \end{itemize}
\end{description}

A scenario where !policy! found that there is nothing to build is worked out
in scenario of figure~\ref{fig:share000.cppmake.cppmake_test_executor_nothing_to_build_uml_sd}.

\begin{figure}[tbh]
    \begin{center}
        \includegraphics{\procts share000/tgt/fig/cppmake_test_executor_nothing_to_build_uml_sd.eps}
    \end{center}
    \caption{cppmake executor nothing to build (UML sequence diagram) }
	\label{fig:share000.cppmake.cppmake_test_executor_nothing_to_build_uml_sd}
\end{figure}

%------------------------------------------------------------------------------
\section{IDE filter}
\label{sec:share000.cppmake.ide_filter}
%------------------------------------------------------------------------------

The IDE filter is responsible for:
\begin{itemize}
\item Converting a generic output message from a makeable into a message 
	  that can be parsed by a specific IDE. When any message which includes a
	  file path and line number is selected, the IDE can show the related source.
\item IDE filter shall support the following message types:
    \begin{itemize}
	\item error
    \item warning
    \item note
    \end{itemize}
\item Upon request, the IDE filter can create project and solution files which
    resemble the code structure of the stem repository.
\end{itemize}

%------------------------------------------------------------------------------
\section{Process}
\label{sec:share000.cppmake.process}
%------------------------------------------------------------------------------

The !process! component is responsible for:
\begin{itemize}
\item Hosting a general make process
\item Output of the make process shall be forwarded to the !makeable! object
\item The process can be executed and aborted
\item Encapsulate platflorm specific solutions (for forwarding process output)
\end{itemize}

%------------------------------------------------------------------------------
\section{Makeable}
\label{sec:share000.cppmake.makeable}
%------------------------------------------------------------------------------

The !makeable! components are responsible for:
\begin{itemize}
\item Providing specific make processes, such as compile and link
\item Translating process output into logger messages
\item Having full control over all intermediate and target objects. The user
      of the !makeable! object can specify the file path of
	  every intermediate and target object.
\item Providing roll-back when an process is aborted, without any loss of data.
      Roll back of an aborted process implies that the end situation is identical
	  as if the process wasn't started at all.
\end{itemize}

%------------------------------------------------------------------------------
\section{Depgraph}
\label{sec:share000.cppmake.depgraph}
%------------------------------------------------------------------------------

The !depgraph! component is responsible for:
\begin{itemize}
\item Indicating which processes can be started next
\item Storing and retreiving the status of the build. Including aborted builds.
\end{itemize}

%------------------------------------------------------------------------------
\section{Path}
\label{sec:share000.cppmake.path}
%------------------------------------------------------------------------------

The !path! component is responsible for:
\begin{itemize}
\item Holding the path (directory and filename) of every file object
\item Having a file system model of all involved files (This is needed to verify
      whether a developer has modified any intermediate or targetfile without
	  using cppmake)
\end{itemize}

%------------------------------------------------------------------------------
\section{Target}
\label{sec:share000.cppmake.target}
%------------------------------------------------------------------------------

The !target! component is responsible for:
\begin{itemize}
\item Identifying targets
\item Relating targets to host platforms (e.g. Linux, Windows)
\item Relating targets to toolsets (e.g. MS, GCC)
\item Relating targets to target platforms (e.g. 32 bit, 64 bit)
\item Relating targets to configurations (e.g. debug, release)
\end{itemize}
